%%=============================================================================
%% Inleiding
%%=============================================================================

\chapter{\IfLanguageName{dutch}{Inleiding}{Introduction}}
\label{ch:inleiding}

%%De inleiding moet de lezer net genoeg informatie verschaffen om het onderwerp te begrijpen en in te zien waarom de onderzoeksvraag de moeite waard is om te onderzoeken. In de inleiding ga je literatuurverwijzingen beperken, zodat de tekst vlot leesbaar blijft. Je kan de inleiding verder onderverdelen in secties als dit de tekst verduidelijkt. Zaken die aan bod kunnen komen in de inleiding~\autocite{Pollefliet2011}:

%%\begin{itemize}
  %%\item context, achtergrond
  %%\item afbakenen van het onderwerp
  %%\item verantwoording van het onderwerp, methodologie
  %%\item probleemstelling
  %%\item onderzoeksdoelstelling
  %%\item onderzoeksvraag
  %%\item \ldots
%%\end{itemize}

Het geautomatiseerd parsen van systeem logs klinkt als een complexe casus en uit de titel kan men niet reeds het probleemdomein afleiden. Om deze titel en aldus het onderwerp beter te begrijpen moeten we ten eerste bespreken binnen welke context het parsen van systeem logs plaatsvindt. Voordat we het domein gaan toelichten is het handig om eerst toe te lichten wat systeem logs specifiek zijn. Hierop geeft de eerste sectie wat meer informatie. Een tweede term waarop we moeten toelichten is wat het parsen van logs specifiek inhoudt. Hierop wordt dieper ingegaan in de tweede sectie. In de laatste sectie gaan we dan dieper ingaan op het specifieke domein waarop we toespitsen binnen deze bachelorproef. Deze bachelorproef beperkt zich tot systeem logs alsook gaan we het parsen van de logfiles als voorbereiding op anomaliedetectie toepassen. 

\section{Wat is een log?}

Een systeem log is een log file gelinkt aan een bepaald systeem (zoals bv. Windows, Linux, etc.). Een log file is een primaire data source voor observaties. Het is een data file gegenereerd door een computer wat informatie bevat over gebruik patronen, activiteiten en operaties die plaatsvinden binnen een operating system, een applicatie, een server of een apparaat. Log files worden gegenereerd wanneer specifieke handelingen gebeuren over het netwerk of binnen de applicatie. Organisaties binnen de IT sector gebruiken logfiles om het gebruik en de activiteiten van hun gebruikers te analyseren alsook voor het gedrag van de applicatie vast te leggen voor eventuele verbeteringen of fouten te detecteren. De hoofdreden waarom log files bestaan is omdat developers van software en hardware het makkelijker vinden om via tekstuele data de werking van het systeem te volgen en vast te leggen. Voor elk domein binnen IT zijn er verschillende logs, voor security, servers, applicaties, etc. Daarbovenop zijn de logs ook nogmaals verschillend van bedrijf tot bedrijf (Google, Apple, Microsoft, etc.).

\section{Log parsing}

Log parsing is een methode die wordt toegepast bij het analyseren van log files. Voordat men een logfile kan analyseren moet men eerst de logfile kunnen opdelen in verschillende delen en deze delen van een label voorzien. Dit is waarvoor men parsers gebruikt. Log parsing is vooral belangrijk om een logfile goed te kunnen analyseren want door logparsing toe te passen kan men specifieke data uit de logs, zoals timestamp, hostname, service name, etc. \autocite{}, makkelijk afleiden. Log parsing kan op verschillende manieren gebeuren. Hieronder worden enkele methodologieën die van belang zijn binnen het onderzoek van deze bachelorproef opgelijst en verder toegelicht:
\begin{itemize}
    \item Regex, hierbij worden reguliere expressies toegepast om alle velden van een systeem log te ontleden. Reguliere expressies zijn een opeenvolging van symbolen en karakters die een string of patroon waarnaar gezocht wordt binnen een text voorstellen.
    \item Grok, combineert meerdere textpatronen naar iets dat overeenkomt met de logs. Dit om logs op te delen in verschillende delen zodat logs mooi gestructureerd en queryable worden.
    \item Parse operator, custom manieren om logs op te delen. Gebruikt als Grok of Regex geen goeie structuur te weeg brengen.
\end{itemize}
Log parsing methodes worden toegepast op verschillende manieren met verschillende strategieën. Deze strategieën ondersteunen de toepassing van de methodologieën. Hieronder worden de verschillende strategieën binnen het onderzoek van deze bachelorproef toegelicht:
\begin{itemize}
    \item Frequent pattern mining: Deze techniek is gebaseerd op frequente patronen die vaak voorkomen in bepaalde datasets.
    Zo kunnen bijvoorbeeld events in een log gezien worden als constante tokens die vaak voorkomen. De procedure die hiermee gepaard gaat is als volgt:\\
     1) De log data meerdere keren itereren.\\
     2) Opbouwen van frequente items en patronen bij elke iteratie.\\
     3) Groeperen van log messages in verschillende clusters.\\
     4) Het extraharen van event templates bij elke cluster. 
    \item Clustering: Bepaalde logs kunnen voldoen aan eenzelfde opbouw of `template`. Deze logs hebben een gelijkaardige vorm. Uit logfiles kan men meerdere vormen halen en deze definiëren in een eindig aantal clusters. Elke log file wordt onderzocht en als de logfile voldoet aan het template van één van deze clusters dan wordt de logfile toegevoegd aan deze cluster. Zo niet, dan wordt op basis van de logfile een nieuwe cluster aangemaakt.
    \item Heuristieken: Logfiles hebben unieke karakteristieken waardoor men heuristieken kan toepassen binnen de logparsing. Dit kan op verschillende manieren:\\
    1) Het verdelen in meerdere groepen van logfiles op basis van het aantal voorkomens van constante en variabele tokens.\\
    2) Door te partitioneren op basis van de lengte van de log messages, de token positie, i.e. waar in de log de message plaatsvindt, en mapping relaties.\\
    3) Met een diepte-gelimiteerde boom de relaties vastleggen en partitioneren.
\end{itemize}
Andere strategieën bestaan nog maar worden niet verder toegelicht in het onderzoek van deze bachelorproef.

\section{Anomalie detectie}

Anomaliedetectie is een moeilijk begrip omdat het een breed onderwerp is. In het algemeen is anomaliedetectie het vinden van patronen binnen de data van een bepaald systeem of een bepaalde applicatie die niet voldoet aan het verwachte gedrag. Deze patronen worden anomalieën genoemd \autocite{chandola2009anomaly}. Men doet aan anomaliedetectie om applicaties en systemen zo performant en betrouwbaar mogelijk te houden. Voorbeelden hiervan zijn fraude detectie binnen bank applicaties (credit cards, etc.), indringers detectie binnen cybersecurity en zelfs binnen militaire surveillance voor vijanden op te sporen \autocite{chandola2009anomaly}. Als er zich een bepaalde fout voordoet bij het gebruik van een systeem of applicatie dan kan dit leiden tot het verlies van gevoelige data. Een fout binnen het verkeer van een systeem kan ook een aanduiding zijn dat er een hacker het systeem is binnengedrongen \autocite{chandola2009anomaly}. Dit is de reden dat men probeert zo goed mogelijk fouten te voorkomen en zo vroeg mogelijk te spotten. Anomaliedetectie wordt toegepast in verschillende domeinen en er zijn ook verschillende technieken ontwikkeld, sommige domein specifiek andere generiek, om aan anomaliedetectie te doen. Om fouten binnen een applicatie of systeem te detecteren maakt men gebruik van log files. Log files zijn tekst bestanden die de werking van een applicatie of systeem vastleggen. Deze worden gegenereerd bij het opstarten van de applicatie en leggen het gedrag van het systeem of de applicatie vast. Door deze te analyseren kan men een perfect beeld krijgen van de doorloop van het systeem proces.

\section{\IfLanguageName{dutch}{Probleemstelling}{Problem Statement}}
\label{sec:probleemstelling}

%%Uit je probleemstelling moet duidelijk zijn dat je onderzoek een meerwaarde heeft voor een concrete doelgroep. De doelgroep moet goed gedefinieerd en afgelijnd zijn. Doelgroepen als ``bedrijven,'' ``KMO's,'' systeembeheerders, enz.~zijn nog te vaag. Als je een lijstje kan maken van de personen/organisaties die een meerwaarde zullen vinden in deze bachelorproef (dit is eigenlijk je steekproefkader), dan is dat een indicatie dat de doelgroep goed gedefinieerd is. Dit kan een enkel bedrijf zijn of zelfs één persoon (je co-promotor/opdrachtgever).

Dit onderzoek zal een meerwaarde vormen voor enkele personen. Als eerste zal de opdrachtgever van dit onderzoek een meerwaarde uit deze bachelorproef halen als informatiebron. Deze bachelorproef zal voor hem een verduidelijking betekenen binnen het onderwerp van logparsing en zal hem instaat stellen om duidelijk de juiste logparser te kiezen voor verder onderzoek binnen anomaliedetectie. Alsook zal dit onderzoek een meerwaarde vormen voor het bedrijf Exitas, die heeft meegedragen als co-promotor binnen deze bachelorproef. Dit bedrijf hanteert op dit moment nog een manuele manier van het parsen van systeem logs. Dit onderzoek zal hun instaat stellen om een duidelijk overzicht te benuttigen van de huidige logparsers en hun specificaties. 


\section{\IfLanguageName{dutch}{Onderzoeksvraag}{Research question}}
\label{sec:onderzoeksvraag}

%%Wees zo concreet mogelijk bij het formuleren van je onderzoeksvraag. Een onderzoeksvraag is trouwens iets waar nog niemand op dit moment een antwoord heeft (voor zover je kan nagaan). Het opzoeken van bestaande informatie (bv. ``welke tools bestaan er voor deze toepassing?'') is dus geen onderzoeksvraag. Je kan de onderzoeksvraag verder specifiëren in deelvragen. Bv.~als je onderzoek gaat over performantiemetingen, dan 

Bij dit onderzoek wordt de vraag gesteld: 'Wat is de huidige snelste log parser die een duidelijke en analyseerbare parsing opleverd voor loganomaliedetectie en die ongesuperviseerd getraind kan worden in een real-time setting?'.

Een gelijkaardig onderzoek is reeds gevoerd in de paper `Tools and Benchmarks for Automated Log Parsing` \autocite{TBA2019}. Dit onderzoek is echter al 3 jaar oud en binnen dit onderzoek gaan we ons toeleggen op andere beperkingen, die reeds binnen de onderzoeksvraag werden vermeld.

Het onderzoek zal ten eerste beperkt worden tot een real-time setting, dit omdat deze bachelorproef een basis voor de analyse en anomaliedetectie van logfiles zal vormen. Hierbij is tijd een belangrijke factor. 

Andere beperkingen die in het onderzoek zullen opgelegd worden, zijn:
\begin{itemize}
    \item Een parser moet een goede parsing opleveren, i.e. een leesbare en duidelijk analyseerbare parsing.
    \item Een gekozen parser moet ‘ongesuperviseerd’ kunnen getraind worden.
    \item Een parser moet snel werken. Hiermee wordt een tijdscomplexiteit bedoeld die hoogstens lineair is in het aantal karakters en aantal verschillende parserboodschappen.
\end{itemize}

\section{\IfLanguageName{dutch}{Onderzoeksdoelstelling}{Research objective}}
\label{sec:onderzoeksdoelstelling}

%%Wat is het beoogde resultaat van je bachelorproef? Wat zijn de criteria voor succes? Beschrijf die zo concreet mogelijk. Gaat het bv. om een proof-of-concept, een prototype, een verslag met aanbevelingen, een vergelijkende studie, enz.

Deze bachelorproef is in het algemeen een vergelijkende studie van de verschillende logparsers. In eerste instantie gaat binnen deze bachelorproef onderzocht worden of de observaties van de paper `Tools and Benchmarks for Automated Log Parsing` \autocite{TBA2019} nog steeds van toepassing zijn. Hieropvolgend gaan de 13 log parsers van de voorgenoemde paper onderhevig gesteld worden aan nieuwe datasets. Als laatste stap gaan er extra logparsers toegevoegd worden aan het onderzoek die niet reeds gedefinieerd zijn in de voorgenoemde paper. Hieruit zal blijken of Drain, i.e. de logparser die in de paper `Tools and Benchmarks for Automated Log Parsing` \autocite{TBA2019} naar bovenkwam als de `beste` en meest efficiënte logparser, ook binnen deze beperkingen en binnen dit onderzoek zal naar bovenkomen als de optimale keuze of dat een andere logparser er bovenuit blinkt. Deze bachelorproef zal naast dit onderzoek ook een duidelijk uiteenzetting maken van het gebruik van logparser \autocite{TBA2019}, met de optimale logparser als voorbeeld. 

\section{\IfLanguageName{dutch}{Opzet van deze bachelorproef}{Structure of this bachelor thesis}}
\label{sec:opzet-bachelorproef}

% Het is gebruikelijk aan het einde van de inleiding een overzicht te
% geven van de opbouw van de rest van de tekst. Deze sectie bevat al een aanzet
% die je kan aanvullen/aanpassen in functie van je eigen tekst.

De rest van deze bachelorproef is als volgt opgebouwd:

In Hoofdstuk~\ref{ch:stand-van-zaken} wordt een overzicht gegeven van de stand van zaken binnen het onderzoeksdomein, op basis van een literatuurstudie.

In Hoofdstuk~\ref{ch:methodologie} wordt de methodologie toegelicht en worden de gebruikte onderzoekstechnieken besproken om een antwoord te kunnen formuleren op de onderzoeksvragen.

% TODO: Vul hier aan voor je eigen hoofstukken, één of twee zinnen per hoofdstuk

In Hoofdstuk~\ref{ch:conclusie}, tenslotte, wordt de conclusie gegeven en een antwoord geformuleerd op de onderzoeksvragen. Daarbij wordt ook een aanzet gegeven voor toekomstig onderzoek binnen dit domein.