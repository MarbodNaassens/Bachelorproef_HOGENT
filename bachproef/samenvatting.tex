%%=============================================================================
%% Samenvatting
%%=============================================================================

% TODO: De "abstract" of samenvatting is een kernachtige (~ 1 blz. voor een
% thesis) synthese van het document.
%
% Deze aspecten moeten zeker aan bod komen:
% - Context: waarom is dit werk belangrijk?
% - Nood: waarom moest dit onderzocht worden?
% - Taak: wat heb je precies gedaan?
% - Object: wat staat in dit document geschreven?
% - Resultaat: wat was het resultaat?
% - Conclusie: wat is/zijn de belangrijkste conclusie(s)?
% - Perspectief: blijven er nog vragen open die in de toekomst nog kunnen
%    onderzocht worden? Wat is een mogelijk vervolg voor jouw onderzoek?
%
% LET OP! Een samenvatting is GEEN voorwoord!

%%---------- Nederlandse samenvatting -----------------------------------------
%
% TODO: Als je je bachelorproef in het Engels schrijft, moet je eerst een
% Nederlandse samenvatting invoegen. Haal daarvoor onderstaande code uit
% commentaar.
% Wie zijn bachelorproef in het Nederlands schrijft, kan dit negeren, de inhoud
% wordt niet in het document ingevoegd.

\IfLanguageName{english}{%
\selectlanguage{dutch}
\chapter*{Samenvatting}
%\lipsum[1-4]
\selectlanguage{english}
}{}

%%---------- Samenvatting -----------------------------------------------------
% De samenvatting in de hoofdtaal van het document

\chapter*{\IfLanguageName{dutch}{Samenvatting}{Abstract}}

%\lipsum[1-4]
Met de hoeveelheid aan applicaties die vandaag aanwezig zijn is het analyseren van log bestanden essentieel geworden. Log bestanden houden data over de werking van de applicatie bij. Door deze te analyseren kan de werking van een applicatie achterhaald worden en kunnen problemen voorkomen worden. Deze logbestanden komen echter in verschillende vormen en in grote hoeveelheden voor. Om log bestanden te kunnen analyseren moeten deze een eenduidige en leesbare structuur bevatten en daarom is het belangrijk om deze te parsen. Het parsen van logbestanden zorgt voor een goede structuur en door het bijhouden van de verschillende clusters waartoe een logbestand kan behoren kunnen clusters met errors bekeken en onderzocht worden.\\

Deze bachelorproef geeft een overzicht van verschillende parsers die nu online beschikbaar zijn en bouwt verder op de paper `Tools and Benchmarks for Automated Log Parsing` \autocite{TBA2019} met twee nieuwe parsers, twee nieuwe datasets en een nieuwe parser als testmethode. Dit voorgaand onderzoek was reeds enkele jaren oud en daarom is er besloten om met deze bachelorproef dit vorig onderzoek opnieuw uit te voeren om zo recentere resultaten te bekomen. Hierbij was het belangrijk om ook te kijken naar nieuwe parsers die sinds het vorige onderzoek zijn uitgekomen en om nieuwe datasets te introduceren. Het vorig onderzoek werkte ook met technologie die sinds dan vernieuwd is en daarom gebruikt dit onderzoek de nieuwste versies van de verscheidene technologiën om zo een duidelijk beeld te geven over waar dit onderzoek nu staat.\\

Hiernaast zal dit onderzoek een begin vormen voor anomaliedetectie op basis van log bestanden. Zoals eerder gezegd kunnen log bestanden gebruikt worden om problemen binnen een applicatie te achterhalen. In een samenleving die constant met applicaties bezig is is het belangrijk om deze altijd operationeel te houden en de kans op problemen te verminderen.\\

Binnen deze bachelorproef zijn er verschillende parsers onderzocht geweest op basis van verschillende criteria: De snelheid, accuraatheid, f-score, de gevonden clusters. Deze parsers zijn onderhevig gesteld aan verschillende datasets en hun resultaten werden vergeleken om zo de parser met de beste resultaten te kunnen onderscheiden en een antwoord te vinden op de onderzoeksvraag, i.e.\ Wat is de huidige snelste log parser die een duidelijke en analyseerbare parsing oplevert voor loganomaliedetectie en die ongesuperviseerd getraind kan worden in een real-time setting?.\\

In deze bachelorproef wordt een duidelijke uiteenzetting van de verschillende parsers gegeven zodat de werking duidelijker wordt. Hiernaast werden de parsers zoals hierboven vermeld, vergeleken met elkaar op verschillende datasets om uiteindelijk een parser te vinden die het best gebruikt zou kunnen worden in een real-time setting voor anomaliedetectie. Dus een parser die op een snelle manier een leesbare en gestructureerde parsing weer kan geven. Dit onderzoek en een groot deel van de resultaten zijn te vinden op de GitHub repository van deze bachelorproef \footnote{https://github.com/MarbodNaassens/Bachelorproef\_HOGENT.git}.\\

Het resultaat was dat de parser die in de vorige paper naarbovenkwam als de meest performante parser hier ook weer uitgekomen is als de meest performante parser die ongesuperviseerd werkt en een goede parsing levert. Zelfs op nieuwe datasets en met nieuwe parsers kwam deze parser de meest consistente en duidelijke parsings uit.
De belangrijkste conclusie die hieruit voortkwam was dus dat deze parser, i.e. Drain, het best zou benut worden binnen een real-time anomaliedetectie setting. Een nauwe opvolger hiervan is echter NuLog en door het onderzoek gevoerd met deze parser is gebleken dat deze mits enige extra aanpassingen ook een goede oplossing kan bieden.\\

Hierdoor blijft de vraag nog open of deze parser een betere oplossing kan vormen als de werking ervan aangepast wordt zodat deze minder grote aanpassingen moet uitvoeren voor verschillende datasets. Voor verder onderzoek zou er ook nog gekeken kunnen worden naar Logram, deze parser was zeer snel maar gaf niet altijd een goede parsing terug waardoor deze achterop geraakte. Bij een aanpassing kan deze misschien ook benut worden binnen dezelfde setting.\\

Verder onderzoek moet ook nog gevoerd worden in verband met het gebruik van Drain binnen anomaliedetectie omdat deze inwerking buiten de scope van deze bachelorproef was gerekend.