%%=============================================================================
%% Voorwoord
%%=============================================================================

\chapter*{\IfLanguageName{dutch}{Woord vooraf}{Preface}}
\label{ch:voorwoord}

%% TODO:
%% Het voorwoord is het enige deel van de bachelorproef waar je vanuit je
%% eigen standpunt (``ik-vorm'') mag schrijven. Je kan hier bv. motiveren
%% waarom jij het onderwerp wil bespreken.
%% Vergeet ook niet te bedanken wie je geholpen/gesteund/... heeft

Dit onderwerp was iets wat mij persoonlijk aansprak omdat ik zelf reeds in aanraking ben gekomen met log bestanden. Als er problemen zijn binnen een programma dat ik gebruik dan ga ik hiernaar kijken om er wijzer uit te worden of om deze mee te geven aan de klantenservice. Hierbij geraak ik zelf niet altijd uit aan de betekenis van de logs en zelfs de klantenservice van grote bedrijven weten hier niet altijd raad mee. Daarom leek het onderzoeken van algoritmes die deze bestanden leesbaarder maakten een leuk bachelorproef onderwerp.\\

Persoonlijk wist ik zelf niet veel over de wereld van log parsers en daarom vond ik dit ook een goede leerervaring. Ik heb veel bijgeleerd bij het onderzoek, maar omdat dit veel verschillende papers inhield en deze niet altijd even duidelijk waren voor iemand die losstaat van de logging wereld was dit wel niet altijd even makkelijk. Het was wel een leuke ervaring en ik hoop dat deze bachelorproef een meerwaarde kan vormen voor mijn promotor en co-promotor zodat ze dit in hun toekomstig werk kunnen benutten.\\

Deze bachelorproef zou nooit tot stand gekomen zijn zonder hulp van verschillende personen. Deze wil ik hieronder bedanken:
\begin{itemize}
    \item Christina Christiaens: Eerst en vooral wil ik mijn vriendin bedanken voor al haar steun en geduld tijdens de maanden van mijn bachelorproef en voor het nalezen van mijn bachelorproef.
    \item Stijn Lievens: Mijn promotor wil ik bedanken voor de consistente en duidelijke opvolging. Ik kon voor al mijn vragen bij hem terecht en hij kon mij altijd goed verder helpen, ook wil ik hem bedanken voor het tussentijds evalueren en nalezen van mijn bachelorproef. 
    \item Geert De Paep: Mijn co-promotor wil ik bedanken voor het nalezen van mijn voorstel, het geven van goede raad en de hulp bij het integreren van nieuwe datasets in mijn onderzoek.
    \item Simon De Wilde: Ik wil mijn medestudent Simon bedanken voor het helpen bij problemen met de bachelorproef en voor zijn geduld bij mijn vragen.
    \item Ziggy Moens: Ik wil mijn medestudent Ziggy bedanken voor het nalezen van mijn voorstel, het nalezen van de integratie bij de Stand van Zaken, het geven van raad en het bijstaan bij problemen.
\end{itemize}
